\section{Implementering}
Implementeringen et minimaxtræet med en heustik baseret på captures og positioner. 
Det er udviklet til at virke med spilmodulerne \texttt{board.py} og \texttt{move.py} som repræsenterer brættet og den tilhørende
funktionalitet samt datatypen for et træk.
I dette kapitel redegører jeg for modulet \texttt{minimax.py}.

\subsection{Minimaxtræ}
Minimaxtræet implementeres ved brug af to definerede datatyper \texttt{Node} og \texttt{Tree}.
Disse danner en dedikeret datastruktur til at håndtere spiltræet. I dette afsnit uddyber jeg
valget af funktioner og felter.

\newpage
\subsubsection{Node}
Datatypen \texttt{Node} repræsentere en tilstand i spillet. Det betyder at den
skal indeholde brikkernes positioner og alle spilleres mulige træk. Derudover skal
den have en reference til dens \textit{parent node} som er tilstanden der førte dertil. Samtidig skal
den også selv være \textit{parent node} til alle mulige tilstande den selv føre til givet spilleres
muligheder for at rykke sine brikker. De tilstande den føre til kaldes \textit{child nodes}.\\
\\
Datatypen er implementeret som en class i python. 
\bigbreak
\texttt{Node} har følgende felter:
\begin{itemize}
    \item \texttt{board}\\
    Brættets der indeholer brikkernes positioner implementeret i \texttt{board.py}.
    \item \texttt{parent}\\
    En reference til dens \textit{parent node}.
    \item \texttt{parent\_move}\\
    Fjendens træk der førte til tilstanden implementeret i \texttt{move.py}.
    \item \texttt{moves}\\
    Liste med træk der må foretages fra tilstanden.
    \item \texttt{child\_nodes}\\
    Liste med referencer til de næste tilstande.
    \item \texttt{maximaxing}\\
    Boolsk værdi der indeholder information hvis' spiller tilstandens tur tilhører. True for den spillende spillers tur.
    \item \texttt{value}\\
    Integer der indeholder tilstandes værdi for spilleren. Jo højere, jo bedre.
    \bigbreak
    Og tilhørende funktioner:
    \item \texttt{make\_node(board: Board, parent\_move: Move=(0,0), parent: Node=None, maximaxing: bool=True, value: int=None) -> Node}\\
    Returnere en \texttt{Node} med default værdier for rodnoden.
    \item \texttt{evaluate\_node(node: Node) -> int} \\
    Vurderer fordelagtigheden af en tilstand for den spillende spiller baseret på det fjendes sidste træk. Jo lavere, jo bedre.
    Da vi ønsker at minimere fjendes positive udfald.
    \item \texttt{expand\_node(node: Node) -> None}\\
    Tilføjer alle mulige permutationer af tilstanden til \texttt{child\_nodes} som \texttt{moves} føre til.
    \item \texttt{node\_moves(board: Board) -> list[Move]}\\
    Retunere træk der er captures blandt mulige træk. Hvis de ikke findes retunere den alle mulige træk.
    \item \texttt{\_next\_move(node: Node) -> Move}\\
    Retunere trækket der føre tilstanden fra roden mod \texttt{node}.
    \bigbreak
    Hjælpefunktionerne:
    \item \texttt{\_is\_capturing(move: Move) -> bool}\\
    Retunere en boolsk værdi om et træk er en capture.
    \item \texttt{\_is\_cornering(move: Move) -> bool}\\
    Retunere en boolsk værdi om et træk er til et hjørne.
\end{itemize}

\newpage
\subsubsection{Tree}
Datatypen \texttt{Tree} repræsenterer spiltræet som starter ved roden og ender i \textit{bladende}.
\textit{Bladene} er alle mulige tilstande et bestemt antal træk fra den nuværende tilstand svarende til
den valgte dybde.
\\
\\
Datatypen er også implementeret som en class i python. Alle tilstande bruger instances af \texttt{Node}.
\bigbreak
\texttt{Tree} har følgende felter:
\begin{itemize}
    \item \texttt{root}\\
    Reference til den node der repræsentere spillets nuværende tilstand.
    \item \texttt{leafes}\\
    Liste med bladene.
    \bigbreak
    Og tilhørende funktioner:
    \item \texttt{make\_tree(board: Board, height: int) -> Tree}\\
    Returnere et \texttt{Tree} med dybden \texttt{height}.
    \item \texttt{contruct\_tree(node: Node, height: int, tree: Tree, acc: int=0) -> None}\\
    Laver træet og vurdere bladene baseret på heustikken. Bladene tilføjes til \texttt{tree.leafes}.
    \item \texttt{find\_max(tree) -> Node}\\
    Retunere bladet med den største værdi.
\end{itemize}

\subsection{Algoritme}
Implementationen af minimax algoritmen bruger en kombination af programmeringsteknikker til at opnå dette.
Træet konstrueres i funktionen \texttt{contruct\_tree} rekursivt for for hver node og dens \textit{child nodes}
og bivirkninger tilføjer bladene til træet i \texttt{leafes}. Undervejs akkumeleres værdierne af 
alle nodes langs stien fra \textit{roden} til \textit{bladet} hvor den tildeles.\\
\\
\texttt{expand\_node} tilføjer iterativt de næste tilstande til en parent for træk tilstanden
tillader. Hvert træk simuleres på en kopi af brættet, hvorefter det nye bræt, trækket og parent noden
selv danner de nye tilstande som tilstanden føre til. Max og min nodes bestemmes også her -- child nodes
er det modsatte af deres parent.\\
\\
\texttt{node\_moves} sortere i de mulige træk. Det giver færre forgreninger hvilket betyder færre rekursive kald.
Derudover motivere det også autospilleren til at capture hvilket igen simplificere evalueringsfunktionen.
Ulempen er at der er tilfælde hvor angreb ikke er den mest optimale taktik.\\
\\
\texttt{find\_max} bruger reduce og finder bladet med den største værdi. Stien fra bladet til
roden følges rekursivt af \texttt{\_next\_move} som returnere trækket fra roden der har en sti
mod bladet.

\subsection{Heustik}


% \subsubsection{Generelt om inputtet}
% Input til programmet består af tal der hører til besteme valg vist i terminalen. 
% Denne beslutning er taget for at minimere muligheden for indstastningsfejl
% og simplificere interaktionen for brugeren.
% Kort sagt bliver brugeren præsentereret for nogle valg og vælger ved 
% at indtaste det tal der hører til valget.

% \subsubsection{Generelt om outputtet}
% Output til brugeren vises formateret i terminalen. Det er altsammen indrykket en tab
% for at skabe er 'polstret' indtryk af brættet. Brættet samt den nuværende spiller 
% vises på skærmen inden hver tur, hvilket sikre at en opdateret version vises og
% at man kan se hvem der har tur. Efter hver tur vises både spiller og træk, og om det er
% en person eller computeren der har rykket.
% \subsubsection{Tilstande}
% Brugeren får valget mellem fem muligheder i menuen, de fire måder to farver kan 
% kombineres med to spillere, samt en mulighed for at afbryde programmet.
% \newline
% Menuen vises ved at kalde \texttt{print\_menu}, hvorefter \texttt{get\_mode} kaldes som retunere informationen til programmet.
% Dette valg har betydning for programmets kontrol flow i og med vi skal have input fra spilleren.
% Denne information gemmes i variablen \texttt{game\_mode} og kan ændres undervejs.

% \subsubsection{Tur}
% Under spillet skal programmet spørge spilleren om at specificere sit træk, ellers skal computeren
% vælge et træk. 
% Dette opnås baseret på værdien i \texttt{game\_mode} i. Hjælpefunktionen \texttt{\_ai\_plays}
% er vedhæftet i Bilag \ref{code:kildekode}.
% \bigbreak
% Programmet spørger først spilleren om at specificere sit træk, eller ændre spillets tilstand:
% \begin{itemize}
%     \item Hvis man vælger at specifiere trækket, vises mulige træk, hvorefter \texttt{get\_move} kaldes.
%     \item Hvis man vælger at ændre tilstanden, vises menuen, hvorefter \texttt{get\_mode} kaldes.
% \end{itemize}
% Når det er computerens tur, bruges funktionen \texttt{next\_move} fra interfacet. Brættet opdateres hver gang
% ved at benytte \texttt{move}.

% \subsubsection{Visning af brættet}
% Brættet vises vha. af \texttt{board\_list} der konstruere en liste med
% 25 elementer hvor hver plads svare til den korresponderende plads på brættet. Her 
% tages højde for at listen er 0-indekseret.
% Dette valg simplificere funktionen \texttt{print\_board} som viser brættet i terminalen.
% Hjælpefunktionen \texttt{\_board\_list} er vedhæftet i Bilag \ref{code:kildekode}.

% \subsubsection{Slut spil}
% Spillet bryder ud af game loopen, når spillet er slut eller tilstanden er 0. Dvs. når \texttt{is\_game\_over} returnere \texttt{False}
% eller brugeren vælger at afbryde.