\section{Implementering}
Implementeringen et minimaxtræet med en heustik baseret på captures og positioner. 
Det er udviklet til at virke med spilmodulerne \texttt{board.py} og \texttt{move.py} som repræsenterer brættet og den tilhørende
funktionalitet samt datatypen for et træk.
I dette kapitel redegører jeg for modulet \texttt{minimax.py}.

\subsection{Minimaxtræ}
Minimaxtræet implementeres ved brug af to definerede datatyper \texttt{Node} og \texttt{Tree}.
Disse danner en dedikeret datastruktur til at håndtere spiltræet. I dette afsnit uddyber jeg
valget af funktioner og felter.

\newpage
\subsubsection{Node}
Datatypen \texttt{Node} repræsentere en tilstand i spillet. Det betyder at den
skal indeholde brikkernes positioner og alle spilleres mulige træk. Derudover skal
den have en reference til dens \textit{parent node} som er tilstanden der førte dertil. Samtidig skal
den også selv være \textit{parent node} til alle mulige tilstande den selv føre til givet spilleres
muligheder for at rykke sine brikker. De tilstande den føre til kaldes \textit{child nodes}.\\
\\
Datatypen er implementeret som en class i python og kan ses i Bilag \ref{code:node}
\bigbreak
\texttt{Node} har følgende felter:
\begin{itemize}
    \item \texttt{board}\\
    Brættets der indeholer brikkernes positioner implementeret i \texttt{board.py}.
    \item \texttt{parent}\\
    En reference til dens \textit{parent node}.
    \item \texttt{parent\_move}\\
    Fjendens træk der førte til tilstanden implementeret i \texttt{move.py}.
    \item \texttt{moves}\\
    Liste med træk der må foretages fra tilstanden.
    \item \texttt{child\_nodes}\\
    Liste med referencer til de næste tilstande.
    \item \texttt{maximaxing}\\
    Boolsk værdi der indeholder information hvis' spiller tilstandens tur tilhører. True for den spillende spillers tur.
    \item \texttt{value}\\
    Integer der indeholder tilstandes værdi for spilleren. Jo højere, jo bedre.
    \bigbreak
    Og tilhørende funktioner:
    \item \texttt{make\_node(board: Board, parent\_move: Move=(0,0), parent: Node=None, maximaxing: bool=True, value: int=None) -> Node}\\
    Returnere en \texttt{Node} med default værdier for rodnoden.
    \item \texttt{evaluate\_node(node: Node) -> int} \\
    Vurderer fordelagtigheden af en tilstand for den spillende spiller baseret på det fjendes sidste træk. Jo lavere, jo bedre.
    Da vi ønsker at minimere fjendes positive udfald.
    \item \texttt{expand\_node(node: Node) -> None}\\
    Tilføjer alle mulige permutationer af tilstanden til \texttt{child\_nodes} som \texttt{moves} føre til.
    \item \texttt{node\_moves(board: Board) -> list[Move]}\\
    Retunere træk der er captures blandt mulige træk. Hvis de ikke findes retunere den alle mulige træk.
    \item \texttt{\_next\_move(node: Node) -> Move}\\
    Retunere trækket der føre tilstanden fra roden mod \texttt{node}.
    \bigbreak
    Hjælpefunktionerne:
    \item \texttt{\_distance(move: Move) -> int}\\
    Returnere et træks absolutte afstand.
\end{itemize}

\subsubsection{Tree}
Datatypen \texttt{Tree} repræsenterer spiltræet som starter ved \textit{roden} og ender i \textit{bladende}.
\textit{Bladene} er alle mulige tilstande et bestemt antal træk fra den nuværende tilstand svarende til
den valgte dybde.
\\
\\
Datatypen er også implementeret som en class i python og kan ses i Bilag \ref{code:tree}.
Alle tilstande bruger instances af \texttt{Node}.
\bigbreak
\texttt{Tree} har følgende felter:
\begin{itemize}
    \item \texttt{root}\\
    Reference til den node der repræsentere spillets nuværende tilstand.
    \item \texttt{leafes}\\
    Liste med bladene.
    \bigbreak
    Og tilhørende funktioner:
    \item \texttt{make\_tree(board: Board, height: int) -> Tree}\\
    Returnere et \texttt{Tree} med dybden \texttt{height}.
    \item \texttt{contruct\_tree(node: Node, height: int, tree: Tree, acc: int=0) -> None}\\
    Laver træet og vurdere bladene baseret på heustikken. Bladene tilføjes til \texttt{tree.leafes}.
    \item \texttt{find\_max(tree) -> Node}\\
    Retunere bladet med den største værdi.
\end{itemize}

\subsection{Algoritme}
Implementationen af minimax algoritmen bruger en kombination af programmeringsteknikker til at opnå dette.
Træet konstrueres i funktionen \texttt{contruct\_tree} rekursivt for for hver \textit{node} og dens \textit{child nodes}. Undervejs akkumeleres værdierne af 
alle nodes langs stien fra \textit{roden} til \textit{bladet} hvor den tildeles. Her bruges også
bruges bivirkninger til at tilføje bladene til træet i \texttt{leafes}\\
\\
\texttt{node\_moves} sortere i de mulige træk. Det giver færre forgreninger hvilket betyder færre rekursive kald.
Derudover tvinger det også autospilleren til at capture hvilket igen simplificere evalueringsfunktionen.
Ulempen er at der er tilfælde hvor angreb ikke er den mest optimale taktik.\\
\\
\texttt{expand\_node} tilføjer iterativt de næste tilstande til en parent for alle træk tilstanden
tillader (\textit{undtagen de frasorterede}). Hvert træk simuleres på en kopi af brættet, hvorefter det nye bræt, trækket og parent noden
selv danner de nye tilstande som tilstanden føre til. Max og min nodes bestemmes også her -- child nodes
er det modsatte af deres parent.\\
\\
\texttt{find\_max} bruger reduce og finder bladet med den største værdi. Stien fra bladet til
roden følges rekursivt af \texttt{\_next\_move} som returnere trækket fra roden der har en sti
mod bladet.

\subsection{Køretider}
Evalueringsfunktionen \texttt{evaluate\_node} har en køretid på $\mathcal{O}(1)$.
Det er bevist besluttet at benytte bevægede afstande fremfor antallet af hvide og sorte brikker.\\
\\
\texttt{expand\_node} har en køretid på $\mathcal{O}(n)$. Vi simulere et træk for
alle mulige træk.\\
\\
\texttt{find\_max} har en køretid på $\mathcal{O}(n)$ Dette kunne forbedres ved at have en sorteret liste.\\
\\
Køretiden af \texttt{contruct\_tree} er polynomisk og afhænger af antallet af mulige træk for de
enkelte tilstande. Køretiden stiger voldsomt når en tilstand har mange permutationer.

\subsection{Forbedringer}
Der er altid plads til forbedringer. Her er en liste over nogle af de tanker jeg har haft om det.
\begin{itemize}
    \item Autospilleren er langsom fordi et nyt helt træ konstruereres ved hvert træk.
    I stedet kunne roden rykkes og træet udvides.
    \item Implementere dynamiske programmeringsmetoder.
    \item Det er ikke nødvendigt med en liste over blade når et nyt træ laves ved hvert træk.
    \item Autospilleren kan forbederes ved at undgå at sortere trækkene.
    I stedet kunne man opstille mere sofistikerede regler i evalueringsfunktionen.
    \item Encapsulating. Felterne på class instances burde have metoder til de specifikke formål.
\end{itemize}