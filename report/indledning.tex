\section{Indledning}
Projektets formål i denne del er at udvikle et modul for autospilleren der kan benyttes af spilmodulet \texttt{alquerquer.py} 
udvilket i første fase. I det følgende kapitel fastlægges projektets formål og problemformulering,
som danner grundlaget for projektet. Rapporten afspejler og dokumenterer det udførte projektarbejde på 1. semester for 
projektets tredje fase i kurset DM574 "Introduktion til programmering".

\subsection{Problemformulering}
Formålet med projektet er at undersøge, hvordan en autospiller til spilletmodulet \texttt{alquerque.py} kan implementeres
ved brug af minimax algoritmen. Løsningen skal benytte teknikker fra pensum til formålet.

\subsection{Kravspecifikation}
Overordnet så skal modulet minimax indeholde en metode som retunere det bedste træk for næste spiller: 
\texttt{next\_move(b: Board, depth: int) -> Move}. \\
\\
Spiltræet som skal repræsentere spillets mulige tilstande konstureres i to trin.
\bigbreak
Trin 1: Minimaxtræet
\begin{itemize}
    \item Hver node er en tilstand af spillet.
    \item En dybde der begrænser antal træk man vil analysere.
    \item Hver node \texttt{n} er parent til nodes der er spillets tilstand efter mulige træk fra \texttt{n}.
\end{itemize}
\bigbreak
Trin 2: Heuristikken
\begin{itemize}
    \item En node er enten max- eller minnodes. Maxnodes er tilstande med spillerens tur.
    \item Alle nodes får tildelt en værdi på baggrund af om det er max- eller minnodes.
    \item Alle træets blade vurderes på baggrund af heustikken.
    \item Rodnoden er en maxnode.
    \item Værdien af en maxnode er værdien af bladet med den højeste værdi.
\end{itemize}

\subsection{Projektafgrænsning}
Projektet er ikke udviklet til et slutprodukt. Der vil være plads til forbedringer 
og optimeringer.
Hovedsageligt handler projektet om at gøre brug af de teknikker fra stoffet vi har haft med at gøre.
Autospilleren er lavet men ikke begrænset til projektets implementering af Alquerque.