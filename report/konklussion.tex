section{Konklussion}
I denne rapport har jeg undersøgt og redegjordt for hvordan modulet \texttt{alquerque.py} kan implementeres
ved at tage begrænsninger fra undermodulerne i betragning. Der er plads til at gøre det mere udførligt,
men overordnet set er det vigtigste med. F.eks. udfordringen med at vise brættet i terminalen og
hvorfor programmets kontrol flow styres af den valgt tilstand.
\bigbreak
Udfordringer:
\begin{itemize}
    \item En udfordring har været at bruge globale variabler, hvilken havde effekten at programmet blev mere kompleks/svært at forstå. Det kan være en fordel at gøre funktionernes parametre endnu mere specifikke. F.eks. tager nogle funktioner datatypen Board, men bruger ikke brikkernes positioner. Således kan programmet gøres mere forståligt.
    \item En anden udfordring har været navngivningen af variabler. Det skulle være ensartet gennem hele programmet, og alle navnene skulle gøre klart hvad de betyder i deres sammenhæng. Jeg synes det lykkedes rimeligt.
    \item En tredje udfordring, som ikke er relateret til selve projektet, er manglende feedback fra den første del af projektet.
\end{itemize}
Således er det lykkedes at implementere \texttt{alquerque.py}, der lever op til kravende. Rapporten introducerer og forklarer
implementationen.
Nu ser jeg frem til feedback,
så jeg kan gøre det endnu bedre næste gang. Jeg vil gerne takke læseren for opmærksomheden, og håber at det været lige så fedt at læse den
som det har været for mig at lave den.
