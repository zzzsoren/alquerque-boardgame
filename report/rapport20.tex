\documentclass{article}

% Packages
\usepackage[danish]{babel}
\usepackage[utf8]{inputenc}
\usepackage[T1]{fontenc}
\usepackage{geometry}
\usepackage{xcolor}  % Til farver
\usepackage{listings}
\usepackage{caption}  % Til bedre styring af billedtekster
\usepackage{amsfonts}
\usepackage{amsmath, amssymb, amsthm}

\begin{document}
\begin{titlepage}
    \vfill
    \clearpage\thispagestyle{empty}
    \centering
    {\Large Alquerque}\vspace{0.5cm}\\
    {\bfseries Eksamensprojekt DM574\\
    Del 3}\\
    \vfill
    {\bfseries Vejleder:}\\
    {Luís Cruz-Filipe}
    \vfill
    {\bfseries Lavet af:}\\
    {Søren Rosendahl Christensen\\
    soerc23@student.sdu.dk\\
    project group 20}
    \vskip6cm
\end{titlepage}

%%%%%%%%%%%
%%%%%%%%%%%
%%%%%%%%%%%
\section*{Resumé}
Denne rapport handler om hvordan jeg har implementeret minimax algoritmen til at finde det
næste træk for spilleren. Algoritmen skal vælge det bedste træk ud fra alle mulige 
sekvenser af træk et baseret på et valgt antal træk forud. Det bedste næste træk 
har størst sandsynlighed for at vinde fremadrettet.

\section*{Abstract}
This report is about how I have implemented the minimax algorithm to find the 
next move for the player. The algorithm should choose the best move based on all possible 
sequences of moves based on a chosen number of moves ahead. The best next move 
has the highest probability of winning in the future.

%%%%%%%%%%%
%%%%%%%%%%%
%%%%%%%%%%%
\newpage
\section*{Forord}
Rapporten er lavet på første semester af uddannelsen computer-science på kurset DM574
'Introduktion til programmering' og er i den forbindelse eksamensprojektets 
tredje del ud af tre og vil være baggrund for det mundtlige forsvar af projektet til januar 2024.
\bigbreak
Det må noteres at som et gruppeprojekt har dette ikke været den største success, hvilket kan
skyldes flere grunde. Jeg har har forsøgt at motivere de to andre til at
deltage og tage initiativ både i forbindelse med det faglige men også til sociale sammenkomster \textit{(for projektets skyld)}.
\clearpage
\tableofcontents

%%%%%%%%%%%
%%%%%%%%%%%
%%%%%%%%%%%
\newpage
\section{Indledning}
Projektets formål i denne del er at udvikle et modul for autospilleren der kan benyttes af spilmodulet \texttt{alquerquer.py} 
udvilket i første fase. I det følgende kapitel fastlægges projektets formål og problemformulering,
som danner grundlaget for projektet. Rapporten afspejler og dokumenterer det udførte projektarbejde på 1. semester for 
projektets tredje fase i kurset DM574 "Introduktion til programmering".

\subsection{Problemformulering}
Formålet med projektet er at undersøge, hvordan en autospiller til spilletmodulet \texttt{alquerque.py} kan implementeres
ved brug af minimax algoritmen. Løsningen skal benytte teknikker fra pensum til formålet.

\subsection{Kravspecifikation}
Overordnet så skal modulet minimax indeholde en metode som retunere det bedste træk for næste spiller: 
\texttt{next\_move(b: Board, depth: int) -> Move}. \\
\\
Spiltræet som skal repræsentere spillets mulige tilstande konstureres i to trin.
\bigbreak
Trin 1: Minimaxtræet
\begin{itemize}
    \item Hver node er en tilstand af spillet.
    \item En dybde der begrænser antal træk man vil analysere.
    \item Hver node \texttt{n} er parent til nodes der er spillets tilstand efter mulige træk fra \texttt{n}.
\end{itemize}
\bigbreak
Trin 2: Heuristikken
\begin{itemize}
    \item En node er enten max- eller minnodes. Maxnodes er tilstande med spillerens tur.
    \item Alle nodes får tildelt en værdi på baggrund af om det er max- eller minnodes.
    \item Alle træets blade vurderes på baggrund af heustikken.
    \item Rodnoden er en maxnode.
    \item Værdien af en maxnode er værdien af bladet med den højeste værdi.
\end{itemize}

\subsection{Projektafgrænsning}
Projektet er ikke udviklet til et slutprodukt. Der vil være plads til forbedringer 
og optimeringer.
Hovedsageligt handler projektet om at gøre brug af de teknikker fra stoffet vi har haft med at gøre.
Autospilleren er lavet men ikke begrænset til projektets implementering af Alquerque.

%%%%%%%%%%%
%%%%%%%%%%%
%%%%%%%%%%%
\newpage
\section{Implementering}
Implementeringen et minimaxtræet med en heustik baseret på captures og positioner. 
Det er udviklet til at virke med spilmodulerne \texttt{board.py} og \texttt{move.py} som repræsenterer brættet og den tilhørende
funktionalitet samt datatypen for et træk.
I dette kapitel redegører jeg for modulet \texttt{minimax.py}.

\subsection{Minimaxtræ}
Minimaxtræet implementeres ved brug af to definerede datatyper \texttt{Node} og \texttt{Tree}.
Disse danner en dedikeret datastruktur til at håndtere spiltræet. I dette afsnit uddyber jeg
valget af funktioner og felter.

\newpage
\subsubsection{Node}
Datatypen \texttt{Node} repræsentere en tilstand i spillet. Det betyder at den
skal indeholde brikkernes positioner og alle spilleres mulige træk. Derudover skal
den have en reference til dens \textit{parent node} som er tilstanden der førte dertil. Samtidig skal
den også selv være \textit{parent node} til alle mulige tilstande den selv føre til givet spilleres
muligheder for at rykke sine brikker. De tilstande den føre til kaldes \textit{child nodes}.\\
\\
Datatypen er implementeret som en class i python. 
\bigbreak
\texttt{Node} har følgende felter:
\begin{itemize}
    \item \texttt{board}\\
    Brættets der indeholer brikkernes positioner implementeret i \texttt{board.py}.
    \item \texttt{parent}\\
    En reference til dens \textit{parent node}.
    \item \texttt{parent\_move}\\
    Fjendens træk der førte til tilstanden implementeret i \texttt{move.py}.
    \item \texttt{moves}\\
    Liste med træk der må foretages fra tilstanden.
    \item \texttt{child\_nodes}\\
    Liste med referencer til de næste tilstande.
    \item \texttt{maximaxing}\\
    Boolsk værdi der indeholder information hvis' spiller tilstandens tur tilhører. True for den spillende spillers tur.
    \item \texttt{value}\\
    Integer der indeholder tilstandes værdi for spilleren. Jo højere, jo bedre.
    \bigbreak
    Og tilhørende funktioner:
    \item \texttt{make\_node(board: Board, parent\_move: Move=(0,0), parent: Node=None, maximaxing: bool=True, value: int=None) -> Node}\\
    Returnere en \texttt{Node} med default værdier for rodnoden.
    \item \texttt{evaluate\_node(node: Node) -> int} \\
    Vurderer fordelagtigheden af en tilstand for den spillende spiller baseret på det fjendes sidste træk. Jo lavere, jo bedre.
    Da vi ønsker at minimere fjendes positive udfald.
    \item \texttt{expand\_node(node: Node) -> None}\\
    Tilføjer alle mulige permutationer af tilstanden til \texttt{child\_nodes} som \texttt{moves} føre til.
    \item \texttt{node\_moves(board: Board) -> list[Move]}\\
    Retunere træk der er captures blandt mulige træk. Hvis de ikke findes retunere den alle mulige træk.
    \item \texttt{\_next\_move(node: Node) -> Move}\\
    Retunere trækket der føre tilstanden fra roden mod \texttt{node}.
    \bigbreak
    Hjælpefunktionerne:
    \item \texttt{\_is\_capturing(move: Move) -> bool}\\
    Retunere en boolsk værdi om et træk er en capture.
    \item \texttt{\_is\_cornering(move: Move) -> bool}\\
    Retunere en boolsk værdi om et træk er til et hjørne.
\end{itemize}

\newpage
\subsubsection{Tree}
Datatypen \texttt{Tree} repræsenterer spiltræet som starter ved roden og ender i \textit{bladende}.
\textit{Bladene} er alle mulige tilstande et bestemt antal træk fra den nuværende tilstand svarende til
den valgte dybde.
\\
\\
Datatypen er også implementeret som en class i python. Alle tilstande bruger instances af \texttt{Node}.
\bigbreak
\texttt{Tree} har følgende felter:
\begin{itemize}
    \item \texttt{root}\\
    Reference til den node der repræsentere spillets nuværende tilstand.
    \item \texttt{leafes}\\
    Liste med bladene.
    \bigbreak
    Og tilhørende funktioner:
    \item \texttt{make\_tree(board: Board, height: int) -> Tree}\\
    Returnere et \texttt{Tree} med dybden \texttt{height}.
    \item \texttt{contruct\_tree(node: Node, height: int, tree: Tree, acc: int=0) -> None}\\
    Laver træet og vurdere bladene baseret på heustikken. Bladene tilføjes til \texttt{tree.leafes}.
    \item \texttt{find\_max(tree) -> Node}\\
    Retunere bladet med den største værdi.
\end{itemize}

\subsection{Algoritme}
Implementationen af minimax algoritmen bruger en kombination af programmeringsteknikker til at opnå dette.
Træet konstrueres i funktionen \texttt{contruct\_tree} rekursivt for for hver node og dens \textit{child nodes}
og bivirkninger tilføjer bladene til træet i \texttt{leafes}. Undervejs akkumeleres værdierne af 
alle nodes langs stien fra \textit{roden} til \textit{bladet} hvor den tildeles.\\
\\
\texttt{expand\_node} tilføjer iterativt de næste tilstande til en parent for træk tilstanden
tillader. Hvert træk simuleres på en kopi af brættet, hvorefter det nye bræt, trækket og parent noden
selv danner de nye tilstande som tilstanden føre til. Max og min nodes bestemmes også her -- child nodes
er det modsatte af deres parent.\\
\\
\texttt{node\_moves} sortere i de mulige træk. Det giver færre forgreninger hvilket betyder færre rekursive kald.
Derudover motivere det også autospilleren til at capture hvilket igen simplificere evalueringsfunktionen.
Ulempen er at der er tilfælde hvor angreb ikke er den mest optimale taktik.\\
\\
\texttt{find\_max} bruger reduce og finder bladet med den største værdi. Stien fra bladet til
roden følges rekursivt af \texttt{\_next\_move} som returnere trækket fra roden der har en sti
mod bladet.

\subsection{Heustik}


% \subsubsection{Generelt om inputtet}
% Input til programmet består af tal der hører til besteme valg vist i terminalen. 
% Denne beslutning er taget for at minimere muligheden for indstastningsfejl
% og simplificere interaktionen for brugeren.
% Kort sagt bliver brugeren præsentereret for nogle valg og vælger ved 
% at indtaste det tal der hører til valget.

% \subsubsection{Generelt om outputtet}
% Output til brugeren vises formateret i terminalen. Det er altsammen indrykket en tab
% for at skabe er 'polstret' indtryk af brættet. Brættet samt den nuværende spiller 
% vises på skærmen inden hver tur, hvilket sikre at en opdateret version vises og
% at man kan se hvem der har tur. Efter hver tur vises både spiller og træk, og om det er
% en person eller computeren der har rykket.
% \subsubsection{Tilstande}
% Brugeren får valget mellem fem muligheder i menuen, de fire måder to farver kan 
% kombineres med to spillere, samt en mulighed for at afbryde programmet.
% \newline
% Menuen vises ved at kalde \texttt{print\_menu}, hvorefter \texttt{get\_mode} kaldes som retunere informationen til programmet.
% Dette valg har betydning for programmets kontrol flow i og med vi skal have input fra spilleren.
% Denne information gemmes i variablen \texttt{game\_mode} og kan ændres undervejs.

% \subsubsection{Tur}
% Under spillet skal programmet spørge spilleren om at specificere sit træk, ellers skal computeren
% vælge et træk. 
% Dette opnås baseret på værdien i \texttt{game\_mode} i. Hjælpefunktionen \texttt{\_ai\_plays}
% er vedhæftet i Bilag \ref{code:kildekode}.
% \bigbreak
% Programmet spørger først spilleren om at specificere sit træk, eller ændre spillets tilstand:
% \begin{itemize}
%     \item Hvis man vælger at specifiere trækket, vises mulige træk, hvorefter \texttt{get\_move} kaldes.
%     \item Hvis man vælger at ændre tilstanden, vises menuen, hvorefter \texttt{get\_mode} kaldes.
% \end{itemize}
% Når det er computerens tur, bruges funktionen \texttt{next\_move} fra interfacet. Brættet opdateres hver gang
% ved at benytte \texttt{move}.

% \subsubsection{Visning af brættet}
% Brættet vises vha. af \texttt{board\_list} der konstruere en liste med
% 25 elementer hvor hver plads svare til den korresponderende plads på brættet. Her 
% tages højde for at listen er 0-indekseret.
% Dette valg simplificere funktionen \texttt{print\_board} som viser brættet i terminalen.
% Hjælpefunktionen \texttt{\_board\_list} er vedhæftet i Bilag \ref{code:kildekode}.

% \subsubsection{Slut spil}
% Spillet bryder ud af game loopen, når spillet er slut eller tilstanden er 0. Dvs. når \texttt{is\_game\_over} returnere \texttt{False}
% eller brugeren vælger at afbryde.

%%%%%%%%%%%
%%%%%%%%%%%
%%%%%%%%%%%
\section{Testning}
Funktioner der retunere værdier har doctests der viser at de virker korrekt og alle
funktioner har docstrings der beskriver hvad de gør. Et eksempel kan ses i Bilag \ref{code:kildekode}.

%%%%%%%%%%%
%%%%%%%%%%%
%%%%%%%%%%%
\newpage
\section{Konklussion}
I denne rapport har jeg undersøg og redegjort for hvordan autospiller modulet \texttt{minimax.py}
kan implementeres. Implementationen lever ikke helt op til forventningerne. F.eks. tildeles
alle nodes ikke værdier, men værdierne akkumeleres istedet under konstruktionen af træet.
Det er en design fejl. Der er ikke taget højde for at at træet skal genbruges.\\
\\
Alle mulige træk bliver heller ikke taget i betragning. Det er et dårligt trade off.\\
\\
Og så måske nogle flere diagrammer i rapporten.\\
\\
Når det er sagt, så har det været en utrolig spændende opgave som jeg har nydt at
arbejde med hen over jul og nytår. Jeg er nogenlunde tilfreds med navngivningen og
læsbarheden af programmet.\\
\\
Således er det lykkedes at implementere \texttt{minimax.py}. Jeg håber at der er
små forbedringer at se (\textit{især i rapporten}). Det har været en fornøjelse.
Tak for opmærksomheden.

%%%%%%%%%%%
%%%%%%%%%%%
%%%%%%%%%%%
\newpage
\section{Bilag}
\definecolor{codegreen}{rgb}{0,0.6,0}
\definecolor{codegray}{rgb}{0.5,0.5,0.5}
\definecolor{codepurple}{rgb}{0.58,0,0.82}
\definecolor{backcolour}{rgb}{0.95,0.95,0.92}

\lstdefinestyle{mystyle}{
    backgroundcolor=\color{backcolour},   
    commentstyle=\color{codegreen},
    keywordstyle=\color{magenta},
    numberstyle=\tiny\color{codegray},
    stringstyle=\color{codepurple},
    basicstyle=\ttfamily\footnotesize,
    breakatwhitespace=false,         
    breaklines=true,                 
    captionpos=b,                    
    keepspaces=true,                 
    numbers=left,                    
    numbersep=5pt,                  
    showspaces=false,                
    showstringspaces=false,
    showtabs=false,                  
    tabsize=2
    }
    \lstset{style=mystyle}

\begin{lstlisting}[language=Python, caption={Datatyper}, label={code:datatyper}]
from dataclasses import dataclass
from functools import reduce
from board import *

def next_move(b: Board, n: int = 3) -> Move:
    """Returns the next move for the autoplayer."""
    tree = make_tree(b, n)
    return _next_move(find_max(tree))


@dataclass
class Node:
board: Board
parent: None         # Node
parent_move: Move
moves: list[Move]
child_nodes: list    # list[Node]
maximizing: bool
value: int

@dataclass
class Tree:
    root: Node
    leafes: list[Node]
\end{lstlisting}
\clearpage
\begin{lstlisting}[language=Python, caption={Funktioner for Node}, label={code:node}]
def make_node(board: Board,
              parent: Node=None,
              parent_move: Move=(0, 0),
              maximizing: bool=True,
              value: int=None) -> Node: 
    """Returns a node representing the state of the game."""
    return Node(board, parent, parent_move, node_moves(board), [], maximizing, value)
\newpage
def node_moves(board: Board) -> list[Move]:
    """Returns the legal capturing moves if any otherwise returns the legal moves."""
    moves = legal_moves(board)
    return [m for m in moves if 6 < abs(m[0] - m[1]) or 3 > abs(m[0] - m[1])] or moves

def expand_node(node: Node) -> None:
    """Adds possible state permutations to a state"""
    for m in node.moves:    
        new_board = copy(node.board)
        move(m, new_board)
        node.child_nodes.append(make_node(new_board, node, m, not node.maximizing))

def evaluate_node(node: Node) -> int:
    """Evaluates the how positive a state is based on the previous move (the enemy move)."""
    if _is_cornering(node.parent_move):
        cornered = 3
    else:
        cornered = 0
    if _is_capturing(node.parent_move):
        captured = 5
    else:
        captured = 10
    
    if node.child_nodes == []:
        return 0
    elif node.maximizing:
        # Fjendens captures er skidt for spilleren
        # Fjendes hjorne positioner er generelt skidt for spilleren
        return - cornered - captured
    else:
        return cornered + captured

def _is_capturing(move: Move) -> bool:
    """Determines if a move is a capturing move."""
    return abs(move[0] - move[1]) >= 8 or abs(move[0] - move[0]) == 2

def _is_cornering(move: Move) -> bool:
    """Determines if a move is a move to a corner."""
    return move[1] == 1 or move[1] == 5 or move[1] == 20 or move[1] == 25

\end{lstlisting}
\clearpage
\begin{lstlisting}[language=Python, caption={Funktioner for Tree}, label={code:tree}]
def make_tree(board: Board,height: int) -> Tree:
    """Initializes a GameTree with a given height to a given board."""
    tree = Tree(make_node(board), [])
    construct_tree(tree.root, tree, height)
    return tree

def construct_tree(node: Node, tree: Tree, height: int, acc: int=-1)  -> None:
    """Builds the heuristic tree of a given height"""
    if height == 0 or node.moves == []:
        node.value = acc+evaluate_node(node)
        tree.leafes.append(node)
    else:
        expand_node(node)
        # Expand subnodes rekursivt
        for i in range(len(node.child_nodes)):
            construct_tree(node.child_nodes[i], tree, height-1, acc+evaluate_node(node.child_nodes[i]))

def find_max(tree) -> Node:
    """Returns the leaf with the maximum accumulated value"""
    return reduce(lambda x,y: x if x.value > y.value else y, tree.leafes)

def _next_move(leaf: Node) -> Move:
    if leaf.parent.parent_move == (0, 0):
        return leaf.parent_move
    else:
        return _next_move(leaf.parent)
\end{lstlisting}
\end{document}
    








